\documentclass{article}
\usepackage{blindtext}
\usepackage[utf8]{inputenc}
\usepackage{amsmath}
\usepackage{graphicx}
\usepackage[]{algorithm2e}

\graphicspath{ {./} }

\title{Tietorakenteet ja algoritmit -harjoitustyö toteutusdokumentti}
\author{Hannes Ihalainen}

\date{\today}
\begin{document}
  \maketitle
  \newpage
  \tableofcontents
  \newpage

  \section{Johdanto}

    Projektissa on toteutettu Ukkosen algoritmi ja toteutettu Ukkosen algoritmilla toimiva ratkaisu 
    Longest common substring -ongelmaan. (LCS = TODO)

    Lisäksi työssä on toteutettu itse Vector (vector.h automaattisesti kasvava taulukko), "FastSet"
    (fastset.h nopea lisäys ja haku) ja Bitset -tietorakenteet.

 \section{Ukkosen algoritmin toteutus}

    Toteutin Ukkosen algoritmin tallentaen edget eri tavoilla. Mittasin eri toteutusten tehokkuutta eri kokoisilla aakkostoilla.

    Tässä taulukko työssä toteutetuista edgejen tallennustavoista .
    \\
    \begin{tabular}{c|c|c|c|c} \hline
                            & \textbf{Vector}& \textbf{Vector järjest,} & \textbf{Taulukko} & \textbf{FastSet} \\ \hline
        Edgen haku          & $O(m)$         & $O(\log m)$              & $O(1)$            & $O(\log^2 m)$     \\ \hline
        Edgen lisäys        & $O(m)$         & $O(m)$                   & $O(1)$            & $O(\log m)$       \\ \hline
        Noden lisäys        & $O(1)$         & $O(1)$                   & $O(m)$            & $O(1)$           \\ \hline
        Kok. aikavaatimus   & $O(n*m)$       & $O(n*m)$                 & $O(n*m)$          & $O(n \log^2 m)$   \\ \hline
        Muistinkäyttö       & $O(n)$         & $O(n)$                   & $O(n*m)$          & $O(n)$           \\
    \end{tabular}
    
    \subsection{Implementaatio-rakenteesta...}
        Ensimmäisenä toteutettu taulukko-implementaatio eroaa muista, koska siinä UkkonenTree on template-luokka. (muissa
        implementaatioissa on siitä kopioitu suurin osa puun perusrakenteesta) Kaikki muut versiot implementoivat
        usuffix.h -headerissa määritellyn UkkonenTree -luokan kukin omalla tavallaan, mikä mahdollistaa mm. eri luokkien 
        testausohjelmien compilaamisen samasta lähdekoodista.
        
        Käytännössä usuffix.h:n toteuttavat implementaatiot kuitenkin käyttävät samaa koodia kaikkialla muualla paitsi
        struct Noden implementaatiossa. Koodin kopioimisen välttämiseksi laitoin yhteisen lähdekoodin tiedostoon (usuffix.cpp).
        Toteutus on hieman epäelegantti. Jokaiselle eri implementaatiolla on oma usuffixnode*.cpp -tiedoston, josta 
        includataan usuffix.cpp. Tämä ratkaisu kuitenkin todennäköisesti paras. Toinen vaihtoehto olisi tietysti ollut
        luoda abstracti Node luokka ja eri luokkia, jotka perivät ja toteuttavat sen. Tämä kuitenkin vaatisi, että
        UkkonenTree -luokan funktiot jollakin tapaa valitsisivat, mitä Node-luokkaa käytetään, mikä vaatisi myös melko
        paljon säätöä, ja kun lopullisissa ohjelmissa UkkonenTree kuitenkin käyttää aina samaa Node-luokan toteutusta,
        tuntuisi turhalta lisätä UkkonenTree -luokkaan tietoa mahdollisuudesta, että Noden implementaatio voisi olla 
        vaihdella.
    
  \section{"FastSetin" aikavaatimus}

    "FastSet" on tietorakenne, jossa säilytetään elementtejä järjestetyissä vektoreissa, joiden koot ovat 2:n potensseja.  
    (en tiedä, mikä on tämän tietorakenteen oikea nimi, jos sillä on sellainen)

    Hakeminen tapahtuu binäärihakemalla jokaisesta vektorista. Vektoreita on enintään $\log_2n$, jolloin binäärihakujen 
    aikavaatimukseksi tulee yhteensä enintään $\sum_{i=0}^{\lfloor \log_2n \rfloor}{\log_2{2^i}}\leq \log_2^2 n$. Keskimäärin
    binäärihakuja tarvitsee tehdä n. puolet tästä, joten hakemisen keskiarvoinen aikavaatimukseksi tulee $O(\log_2^2 n)$.

    Elementin lisääminen toteutetaan käytännössä luomalla ensin setti, jossa on vain lisättävä elementti ja yhdistämällä 
    se alkuperäisen setin kanssa.
    
    Kaksi settiä voi yhdistää käymällä läpi molempien vektorit ja yhdistämällä samankokoiset vektorit (jos molemmissa sen
    kokoinen vektori on täysi). Aikavaatimus vektorien yhdistämiseen on $O(k)$, missä $k$ on uuden vektorin koko. 
    Kokonaisaikavaatimus yhdistämisessä on siis enintään $O(\log_2^2 n+log_2^2 m)$. Kesimimäärin yhdistämisen aikavaatimus 
    on kuitenkin $O(log_2 n + log m)$, kun settien koot ovat $n$ ja $m$.
    \\ \\
    Pseudokoodi yhdistämiselle: \\
    \begin{algorithm}[H]
        \For{}{
            read current\;
            \eIf{understand}{
            go to next section\;
            current section becomes this one\;
            }{
            go back to the beginning of current section\;
            }
        }
        \caption{How to write algorithms}
    \end{algorithm}
    
    \textbf{Todistus elementin lisäämisen keskimääräiselle aikavaatimukselle}

    Olkoon jokainen elementti $i$ $n$-kokoisessa setissä vektorissa jonka koko on $2^k_i$. Tällöin elementti $i$ on ollut mukana
    yhdistämisoperaatiossa $k_i$ kertaa. (Alussa $k_i=0$ ja jokaisessa yhdistämisessä se kasvaa yhdellä) Kokonaisaikavaatimus 
    setin rakentamiseen on tällöin $\sum_{i=1}^n{k_i}$. Koska $k_i\leq \log_2n$, kokonaisaikavaatimus on enintään
    $n\log_2n$, jolloin keskimääräinen aikavaatimus elementin lisäämiselle on $\log_2 n$

  \section{Longest common substring}
    Longest Common Substring on ongelma, jossa halutaan selvittää joukolle merkkijonoja, mikä on pisin kaikille merkkijonoille
    yhteinen substring.
    
    Longest Common Substring-ongelman voi ratkaista suffiksipuulla $O(N*K)$ -ajassa, missä $N$ on merkkijonojen yhteispituus ja
    $K$ merkkijonojen määrä. Toteutin tällaisen LCS-algoritmin.
    
    Ratkaisu
\end{document}
