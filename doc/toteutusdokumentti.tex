\documentclass{article}
\usepackage{blindtext}
\usepackage[utf8]{inputenc}
\usepackage{amsmath}
\usepackage{graphicx}
\usepackage[]{algorithm2e}

\graphicspath{ {./} }

\title{Tietorakenteet ja algoritmit -harjoitustyö toteutusdokumentti}
\author{Hannes Ihalainen}

\date{\today}
\begin{document}
  \maketitle
  \newpage
  \tableofcontents
  \newpage

  \section{Johdanto}

    Projektissa on toteutettu Ukkosen algoritmi suffiksipuun muodostamiseen ja toteutettu Ukkosen algoritmilla toimiva ratkaisu 
    Longest common substring -ongelmaan.

    Lisäksi työssä on toteutettu itse Vector (vector.h dynaaminen taulukko), "FastSet"
    (fastset.h tietorakenne, jossa on nopea elementin lisäys ja haku) ja Bitset (bitset.h) -tietorakenteet.
    
 \newpage
 \section{Ukkosen algoritmin toteutus}

    Ukkosen algoritmi muodostaa suffiksipuun lineaarisessa ajassa merkkijonon kokoon nähden. Implementaationi Ukkosen
    algoritmin ytimestä seuraa hyvin tarkasti Ukkosen paperissa pseudokoodilla kuvattuja funktioita, ja saavutettu
    aikavaatimus on ideaalinen, $O(n)$ aakkoston koon ollessa vakio.
    
    Toteutin Ukkosen algoritmin tallentaen edget eri tavoilla. Kaikki toteutukset ovat merkkijonon pituuden suhteen
    lineaarisia, mutta niiden nopeus vaihtelee aakkoston koon mukaan.

    Tässä taulukko työssä toteutetuista edgejen tallennustavoista. $n$ on merkkijonon pituus ja $m$ aakkoston koko. \\
    (Vector on toteuttamani dynaaminen taulukko. "FastSet" taas on toteuttamani nopeampia lisäys- ja haku -operaatioita
    tukeva tietorakenne.) Näiden lisäksi testauksessa oli vertailua varten std::set:tiä käyttävä implementaatio, jonka
    aikavaatimus $O(n \log m)$).
    \\
    \begin{tabular}{c|c|c|c|c} \hline
                            & \textbf{Taulukko} & \textbf{Vector}& \textbf{Vector järjest.} & \textbf{"FastSet"}        \\ \hline
        Edgen haku          & $O(1)$            & $O(m)$         & $O(\log m)$              & $O(\log^2 m)$             \\ \hline
        Edgen lisäys        & $O(1)$            & $O(1), O(m)*$  & $O(m)$                   & $O(\log m), O(\log^2 m)*$ \\ \hline
        Noden lisäys        & $O(m)$            & $O(1)$         & $O(1)$                   & $O(1)$                    \\ \hline
        Kok. aikavaatimus   & $O(n*m)$          & $O(n*m)$       & $O(n*m)$                 & $O(n \log^2 m)$           \\ \hline
        Kok. tilavaatimus   & $O(n*m)$          & $O(n)$         & $O(n)$                   & $O(n)$                    \\
    \end{tabular}
    * Edgen lisäykset Vectoriin ja FastSettiin olisivat teoriassa $O(1)$ ja $O(\log m)$, mutta koska edgen lisäyksen
    yhteydessä tarkistetaan, onko kyseisellä merkillä alkavaa edgeä jo rakenteessa, aikavaatimukseksi tulee sama 
    kuin edgen haussa. \\
    
    \subsection{Implementaatiosta ja tiedostoista ja rakenteesta...}
        Ensimmäisenä toteutettu taulukko-implementaatio eroaa muista siinä, että sen UkkonenTree on template-luokka (aakkoston
        koko on luokkaan sidottu)

        Kaikki muut versiot implementoivat usuffix.h -headerissa määritellyn UkkonenTree -luokan kukin omalla tavallaan, mikä
        mahdollistaa mm. eri luokkien testausohjelmien compilaamisen samasta lähdekoodista.
        
        Käytännössä usuffix.h:n toteuttavat implementaatiot käyttävät samaa koodia kaikkialla muualla paitsi
        struct Noden implementaatiossa. Koodin kopioimisen välttämiseksi laitoin yhteisen lähdekoodin tiedostoon (usuffix.cpp).
        Toteutus on hieman epäelegantti. Jokaiselle eri implementaatiolla on oma usuffixnode*.cpp -tiedoston, josta 
        includataan usuffix.cpp. Tämä ratkaisu kuitenkin todennäköisesti paras. Toinen vaihtoehto olisi tietysti ollut
        luoda abstracti Node luokka ja eri luokkia, jotka perivät ja toteuttavat sen. Tämä kuitenkin vaatisi, että
        UkkonenTree -luokan funktiot jollakin tapaa valitsisivat, mitä Node-luokkaa käytetään, mikä vaatisi myös melko
        paljon säätöä, ja kun lopullisissa ohjelmissa UkkonenTree kuitenkin käyttää aina samaa Node-luokan toteutusta,
        tuntuisi turhalta lisätä UkkonenTree -luokkaan tietoa mahdollisuudesta, että Noden implementaatio voisi olla 
        vaihdella.
        
  \newpage
  \section{"FastSetin" aikavaatimus}

    "FastSet" on tietorakenne, jossa säilytetään elementtejä järjestetyissä dynaamisissa taulukoissa (käytän niistä
    nimitystä vektori), joiden koot ovat 2:n potensseja. (en tiedä, mikä on tämän tietorakenteen oikea nimi, jos sillä 
    on sellainen. Opin sen eräältä kaveriltani erään algoritmileirin yhteydessä). Käytän tästä tietorakenteen vektorijoukosta
    tässä analyysissä nimitystä vektorisetti. Vektorisetin koolla tarkoitan vektorisetin suurimman vektorin järjestyslukua.
    (Järjestysluvultaan $k$:nnen vektorin koon ollessa $2^{k-1}$). $n$ elementtiä sisältävän vektorisetin koko on siis
    $\lfloor \log_2 n \rfloor$. Esimerkiksi 10 elementtiä sisältävässä vektorisetissä olisi kaksi täytettyä vektoria, joiden 
    koot ovat 2 ja 8, ja vektorisetin koko olisi tällöin 4. (väliin jäävät puuttuvat vektorit voidaan ajatella tyhjiksi, 
    jolloin vektorisetissä olisivat vektorit $1_{tyhja}$, $2_{taysi}$, $3_{tyhja}$ ja $4_{taysi}$.)
    
    Pseudokoodissa merkinnällä $A[i]$ tarkoitetaan vektorisetin $A$ vektoria, jonka koko on $2^i$. Tämä vektori saattaa
    olla tyhjä tai täysi.
    \\ \\
    Hakeminen tietorakenteesta tapahtuu binäärihakemalla jokaisesta vektorista. Elementtien määrän ollessa $n$ vektoreita on 
    enintään $\log_2n$, jolloin binäärihakujen aikavaatimukseksi tulee yhteensä enintään $\sum_{i=0}^{\lfloor \log_2n \rfloor}
    {\log_2{2^i}}\leq \log_2^2 n$. Keskimäärinkin binäärihakuja tarvitsee tehdä n. puolet tästä, joten hakemisen keskimääräinen
    aikavaatimus on $O(\log_2^2 n)$.
    \\ \\
    Elementin lisääminen toteutetaan käytännössä luomalla ensin vektorisetti, jossa on vain lisättävä elementti ja yhdistämällä 
    se alkuperäisen vektorisetin kanssa.
    
    Kaksi vektorisettiä yhdistetään käymällä läpi molempien vektorit pienimmästä lähtien ja yhdistämällä samankokoiset vektorit 
    (jos molemmissa sen kokoinen vektori on täysi). Samalla tyhjennetään yhdistettävän vektorisetin vektorit. Aikavaatimus kahden 
    järjestetyn vektorin yhdistämiseen järjestetty vektori tuottaen on $O(k)$, missä $k$ on uuden vektorin koko. Kokonaisaikavaatimus
    yhdistämisessä voi siis pahimmillaan olla $O(n)$, kun vektorisetissä on yhteensä n elementtiä. Keskimäärin operaatio on kuitenkin
    selvästi nopeampi. Yhden elementin lisäämisen keskimääräinen aikavaatimus on $O(\log_2 n)$, mikä todistetaan alempana.
    
    Pseudokoodi yhdistämiselle:
    \\ \\
    \begin{algorithm}[H]
        carryVector:=\{\}\;
        i:=0\;
        \While{i$<$size of B or carryVector is not empty}{
            \eIf{none of A[i], B[i] and carryVector is empty}{
                carryVector:=merge vectors B[i] and carryVector\;
                B[i]:=\{\}\;
            }{
                \eIf{two of A[i], B[i] and carryVector are not empty}{
                    carryVector:=merge the two\;
                    A[i]:=\{\}\;
                    B[i]:=\{\}\;
                }{
                    swap(A[i], $\max_{size}$ \{B[i], carryVector\})\;
                }
            }
            i:=i+1\;
        }
        \caption{Merge vectorset B to A}
    \end{algorithm}
    
    \medskip \medskip
    \textbf{Todistus elementin lisäämisen keskimääräiselle aikavaatimukselle}
    Pseudokoodissa esitetty swap-operaation oletetaan toimivan $O(1)$ -ajassa. Näin ollen algoritmin aikavaatimukseksi tulee
    $O(\log_2 n + M)$, missä $M$ on suoritettujen yhdistämisoperaatioiden aikavaatimus.
    \\ \\
    Olkoon jokainen elementti $i$ $n$ elementtiä sisältävässä vektorisetissä vektorissa jonka koko on $2^{k_i}$. Tällöin
    elementti $i$ on ollut mukana yhdistämisoperaatiossa $k_i$ kertaa. (Alussa $k_i=0$ ja jokaisessa yhdistämisessä 
    se kasvaa yhdellä) Kokonaisaikavaatimus vektorisetin rakentamisen yhdistämisoperaatioihin on tällöin $\sum_{i=1}^n{k_i}$. 
    Koska $k_i\leq \log_2n$, kokonaisaikavaatimus on enintään $n\log_2n$. Toisaalta yli puolet elementeistä on ollut yhdistämisessä
    mukana maksimimäärän $\lfloor \log_2n \rfloor$, joten kokonaisaikavaatimus on $O(n\log_2n)$ ja keskimääräinen aikavaatimus elementin lisäämiselle 
    on $O(\log_2 n)$.

  \newpage
  \section{Longest Common Substring}
    Longest Common Substring on ongelma, jossa halutaan selvittää joukolle merkkijonoja, mikä on pisin kaikille merkkijonoille
    yhteinen substring.
    
    Longest Common Substring-ongelman voi ratkaista suffiksipuulla $O(N*K)$ -ajassa, missä $N$ on merkkijonojen yhteispituus ja
    $K$ merkkijonojen määrä. Toteutin osana projektia tällaisen LCS-algoritmin.
    
    Ideana on muodostaa merkkijonoista \textit{generalized suffix tree}, eli muodostaa yksi merkkijono, jossa merkkijonot ovat
    peräkkäin ja jokaiselle on lisätty loppuun oma erityismerkkinsä ja luoda sille suffiksipuu.
    Esimerkiksi merkkijonoista $abbabb$, $ccbbabcc$ ja $ababc$ voitaisiin muodostaa merkkijono $abbabb\$_0ccbbabcc\$_1ababc\$_2$
    ja sitten luoda sille suffiksipuu. ($\$_0$, $\$_1$ ja $\$_2$ ovat uniikkeja erityismerkkejä)
    
    Tässä suffiksipuussa jokainen $\$_i$ on varmasti lehteen vievällä edgellä, koska haarautuminen edellyttäisi sitä, että
    haarautumiskohtaa vastaava merkkijono esiintyy substringinä vähintään kaksi kertaa alkuperäisessä merkkijonossa. Haarautuminen
    $\$_i$:n jälkeen ($\$_i$:n alipuussa) tarkoittaisi siis sitä, että olisi ainakin kaksi samanlaista substringiä, joissa merkki 
    $\$_i$ olisi mukana. Tämä on mahdotonta, koska jokainen $\$_i$ on uniikki ja esiintyy merkkijonossa vain kerran. Koska 
    merkkijonon viimeinen merkki on erityismerkki, jokaisella lehteen vievällä edgellä on ainakin yksi erityismerkki.
    
    Jokaista nodea, joka ei ole lehti, vastaa siis substring, jossa ei ole mukana erityismerkkejä. Pisimmän yhteisen substringin
    on oltava jokin näistä nodeja vastaavista substringeistä. (Edgen keskellä olevaa tilaa vastaavaa substringiä voi aina pidentää 
    seuraavaan nodeen, koska edgen keskellä olevaa tilaa ja seuraavaa nodea vastaavilla substringeillä on samat esiintymät 
    alkuperäisessä merkkijonossa. Lisäksi lehdistä ei voi tulla ratkaisuja.)
    
    Nodea vastaava substring on erityismerkkiin $\$_i$ päättyvän merkkijonon substring jos ja vain jos noden alipuussa on lehti, jossa
    $\$_i$ on ensimmäinen erityismerkki.
    
    LCS:ää vastaava tila puussa on siis sellaisen noden kohdalla, jonka alipuusssa on jokaiselle $\$_i$:lle lehti, jossa $\$_i$ 
    on ensimmäinen erityismerkki, ja jota vastaava merkkijono on mahdollisimman pitkä.
    
    Tällaisen noden etsiminen on helppo toteuttaa $O(N*K)$ -ajassa. Ratkaisussa rekursoin puun läpi ($O(N)$ nodea) ja katson 
    jokaisen noden kohdalla $K$-kokoisesta bitsetistä, millä $\$_i$:llä alkavia lehtiä sen noden alipuussa ovat. Tämän jälkeen
    on vain valittava kaikkille merkkijonoille yhteisistä substringeistä pisin/pisimmät, mikä on helppo toteuttaa esim. käymällä
    kaikki yhteisiä substringejä vastaavat nodet läpi.
    
    \newpage
    
    Pseudokoodi rekursiiviselle funktiolle findCommonStrings:
    \\ \\
    \begin{algorithm}[H]
        bitSet:=bitSet of size K initially filled with zeros\;
        \For{each child of node}{
            \eIf{child is leaf}{
                bitSet.setBit(i of first $\$_i$ on edge)\;
            }{
                bitSet$|=$(findCommonStrings(child))\;
            }
        }
        \If{every bit in bitSet is 1}{
            commonStrings.append(node)\;
        }
        \Return{bitSet}
        \caption{findCommonStrings}
    \end{algorithm}

    \medskip \medskip
    Rekursiivinen funktio lisää siis listaan commonStrings kaikki nodet, joita vastaava merkkijono on yhteinen kaikille merkkijonoille.
    
    Koko puun käsittelyn aikavaatimus on selvästi $O(N*K)$.
    
    Projektin LCS-implementaatiossa käytin suffiksipuu-algoritmeista sitä, missä edget tallennetaan Vectoriin. (Se on käytännössä paras,
    kun aakkoston koko on enintään muutama sata merkkiä)
    
    
\end{document}
\grid
